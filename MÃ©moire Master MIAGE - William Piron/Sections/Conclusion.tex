Capter l'attention et transmettre de la meilleure manière possible des connaissances sont des problématiques anciennes et qui se poseront tant qu'il restera du savoir à communiquer à de nouvelles générations. Trouver à la fois une méthode qui convienne au plus grand nombre et soit au moins aussi efficace que les méthodes déjà en application est un défi à relever continuellement et c'est dans cette continuité que ce mémoire cherche à s'inscrire. \par

Par ces activités, ces propositions d'éléments issus de jeux et les méthodes à appliquer à l'encadrement des étudiants, la démarche de ce mémoire est d'apporter sa pierre à l'édifice dans le domaine de l'apprentissage de la Recherche Opérationnelle. L'expérience proposée ici est le résultat de l'application de méthodes existantes et éprouvées à un programme d'enseignement également en pleine réécriture afin de mieux correspondre aux étudiants de la filière MIASHS/MIAGE de l'Université Paris-Nanterre. De meilleurs résultats et une meilleure implication devraient être observés du coté des étudiants et un meilleure retour sur les méthodes utilisées et la construction du cours du coté enseignant. \par

Dans le futur, une fois cette expérience réalisée et des résultats concrets obtenus, il sera possible de vérifier la justesse de notre hypothèse, de modifier et adapter au besoin les composantes du cours afin d'affiner, améliorer et compléter l'enseignement. De même, il serait possible d'appliquer cette gamification moyennant quelques adaptations à d'autres matières enseignées lors du cursus MIASHS/MIAGE, voire dans d'autres cursus.