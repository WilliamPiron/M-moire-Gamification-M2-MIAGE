Ce mémoire s'inscrit en conclusion d'un cycle d'études en Licence puis Master MIAGE au sein de l'Université Paris-Nanterre. Au delà de s'additionner aux diverses évaluations dans la validation des acquis, il a pour but de développer l'esprit de recherche et de synthèse des étudiants au sein d'une démarche de questionnement et d'analyse.\par
Ce document encadré par M. François Delbot, maître de conférences à l'Université Paris-Nanterre, est le fruit de cette démarche.

\section{Motivations}
Années après années, l'enseignement est confronté à un problème récurrent : capter l'attention des élèves et étudiants afin de leur transmettre de la meilleure manière possible des connaissances. Selon le domaine étudié, la composition des groupes d'apprenants et leurs personnalités propres, la méthode à appliquer et les moyens utilisés sont variés, tous comme leurs résultats. \par
Tous les étudiants n'attendent pas la même chose d'un enseignement et ne s'y investissent pas avec la même intensité. Certains doivent être motivés ne serait ce que pour assister aux enseignements quand d'autres ont besoin d'un encadrement ou d'une réalisation concrète et précise. \par
La question d'une méthode d'enseignement permettant de répondre ce besoin de manière globale, adaptée à toutes les catégories d'apprenants et aisément mise en place par les enseignants se pose alors.

\section{Objectif du mémoire}
L'objectif de ce mémoire est de poser la question de la gamification appliquée à l'enseignement, plus particulièrement à celui de la Recherche Opérationnelle dans le cadre de la licence MIASHS parcours MIAGE de l'Université Paris-Nanterre. \par
Nous définirons la gamification et ses principes, le cadre dans lequel nous comptons l'évaluer et dans quels objectifs, avant de d'analyser les résultats de ses applications dans d'autres enseignements et domaines par d'autres équipes et chercheurs. \par
Nous proposerons ensuite une expérience pouvant être réalisée au sein de l'Université afin de comparer l'enseignement concerné avec mise en place de gamification par rapport à l'enseignement dit traditionnel de la matière.